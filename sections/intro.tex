\documentclass{article}
\usepackage[utf8]{vietnam}
\usepackage[english]{babel}

\title{Running Tracking Application Report}
\author{Sơn Đào Thái \\ Võ Hồng Quang \\ Trần Thọ Thăng \\ Nguyễn Đức Tâm \\Vũ Thế Khôi Nguyên \\ Ngô Quý Đăng Tuấn}
\date{}

\begin{document}

\maketitle

\newpage

\section{Introduction}

\subsection{Background}

    Nowadays, people spend less time performing physical labor and taking part in heavy tasks. As a result, maintaining a healthy body requires more exercise. Among various options for exercise, running is one of the most accessible and effective ways to keep fit. 

    However, it's essential to follow well-structured plans to maximize the benefits of running. This has created a growing demand for tools that can optimize running and enhance its effectiveness.

    While several running applications exist, they share significant drawbacks such as intrusive ads, inaccurate data, reliance on an internet connection, and more. These disadvantages lead to a poor user experience and reduce the reliability of these tools.

    To address the problems of other applications, we decide to develop a mobile application that offers a more user-friendly and effective solution.

\subsection{Objectives}

    The Running App is designed to help users track and monitor their outdoor running activities by utilizing GPS data. It tracks the user's running time, distance, speed, and calories burned. These data will be saved for future reference and analyzed to generate appropriate guides. The app aims to provide a user-friendly interface for runners of all levels, making it easy to track their progress.
    Our objectives are:
    \begin{itemize}
        \item Tracking user's location in real time and display within the app
        \item Analyzing run session data to increase the effectiveness
        \item Providing user-friendly interface
    \end{itemize}
\subsection{Overall Approach}

    The development of the running tracker application follows a user
centered approach to ensure a good user experience while working
 with accurate and meaningful data
    We follow the following steps to design and develop the application:
    \begin{itemize}
        \item Understanding User Requirements:\\
        The first step is analyzing the target audience and collecting their feedback. This ensure the app's feature will satisfy users' requirements
        \item System Design and Architecture:\\
        The model used in this project is MVVM (Model-View-ViewModel) desgin pattern. This architecture separates the user interface logic, application logic, and data management to ensure flexibility, maintainability, and scalability.
        \item Key Features and Functionalities:\\
        \begin{itemize}
            \item View Layer: Implement a user-friendly UI using fragments and RecycleViews for display
            \item Filter and Analytics: Use calendar view to select date ranges to display analyse previous run sessions.
            \item Run Tracking: Track user's location and gather data such as distance, speed, time in real time.
            \item Performance Insights: Display visual summaries, trends and personal achievements.
        \end{itemize}
        \item Development Workflow:\\
        The application was developed using Agile methodologies, incorporating iterative cycles of prototyping, testing, and feedback integration. This approach enabled continuous refinement of features.
        \item Testing and Validation:\\
        The application was tested on different Android models with different Android versions and screen size. We also conducted real world testing sessions to validate accuracy and reliability.
    \end{itemize}
        

\subsection{Scope}
    The app will serve runners looking to track their workouts in real-time, providing easy access to essential metrics and an efficient way to save and analyze their performance. It also encourages runner to complete their objective.
    
    Our project includes the following features:
\begin{itemize}
    \item Start/Stop a Run
    \item View Running Stats in Real-Time
    \item Review Previous Runs
    \item Running History
    \item Pause/Resume a Run
    \item GPS-based Tracking
    \item Set Personal Goals
    \item Recommend Training Plans
    \item Show Duration Progress
    \item Notification Alerts
    \item Offline Functionality
    \item Personalization Settings
    \item Real-Time Error Alerts
    \item Notification Break Reminders
    \item Data Deletion
\end{itemize}

\end{document}
