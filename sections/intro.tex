\documentclass{article}
\usepackage[utf8]{vietnam}
\usepackage[english]{babel}

\title{Running Tracking Application Report}
\author{Sơn Đào Thái \\ Võ Hồng Quang \\ Trần Thọ Thăng \\ Nguyễn Đức Tâm \\Vũ Thế Khôi Nguyên \\ Ngô Quý Đăng Tuấn}
\date{}

\begin{document}

\maketitle

\section{Introduction}

\subsection{Background}

    Nowadays, people spend less time performing physical labor and taking part in heavy tasks. As a result, maintaining a healthy body requires more exercise. Among various options for exercise, running is one of the most accessible and effective ways to keep fit. 

    However, it's essential to follow well-structured plans to maximize the benefits of running. This has created a growing demand for tools that can optimize running and enhance its effectiveness.

    While several running applications exist, they share significant drawbacks such as intrusive ads, inaccurate data, reliance on an internet connection, and more. These disadvantages lead to a poor user experience and reduce the reliability of these tools.

    To address the problems of other applications, we decided to develop a mobile application that offers a more user-friendly and effective solution.

\subsection{Objectives}

    The Running App is designed to help users track and monitor their outdoor running activities by utilizing GPS data. It tracks the user's running time, distance, speed, and calories burned. These data will be saved for future reference and analyzed to generate appropriate guides. The app aims to provide a user-friendly interface for runners of all levels, making it easy to track their progress.

\subsection{Overall Approach}

    The development of the running tracker application follows a user-centered approach to ensure good user experience while working with accurate and meaningful data.

\subsection{Scope}

    The app is designed to serve runners looking to track their workouts in real-time, providing easy access to essential metrics and an efficient way to save and analyze their performance. Additionally, it encourages users to complete their objectives and improve their running routines.

\end{document}
